\documentclass[12pt]{article}
\usepackage[margin=1in]{geometry}  % set the margins to 1in on all sides
\usepackage{concrete}
\usepackage{euler}
\usepackage{graphicx}              % to include figures
\usepackage{amsmath}               % great math stuff
\usepackage{amsfonts}              % for blackboard bold, etc
\usepackage{amsthm,amssymb}
\usepackage{mdwtab,booktabs}
\usepackage{pifont}

% various theorems, numbered by section

\newtheorem{thm}{Theorem}[section]
\newtheorem{lem}[thm]{Lemma}
\newtheorem{prop}[thm]{Proposition}
\newtheorem{cor}[thm]{Corollary}
\newtheorem{conj}[thm]{Conjecture}

\DeclareMathOperator{\id}{id}

\newcommand{\bd}[1]{\mathbf{#1}}  % for bolding symbols
\newcommand{\ds}{\displaystyle}      % for Real numbers
\newcommand{\ZZ}{\mathbb{Z}}      % for Integers
\newcommand{\col}[1]{\left[\begin{matrix} #1 \end{matrix} \right]}
\newcommand{\comb}[2]{\binom{#1^2 + #2^2}{#1+#2}}

\begin{document}


\nocite{}

\title{\textbf{The Logarithmic Constant $e$ is  Irrational}}

\author{Miliyon T.\\Addis Ababa University\\ Department of Mathematics}
\maketitle

\section{Characterization of the number $e$.}

\subsection{Using limit}

\begin{prop} The limit
\begin{align}
\lim_{x\rightarrow \infty}\biggl(1+\frac{1}{x}\biggl)^x
\end{align}
converges to $e$.
\end{prop}
\begin{proof} The Limit
\[\lim_{x\rightarrow \infty}\biggl(1+\frac{1}{x}\biggl)^x\]
is indeterminate form of the type $1^{\infty}$. So let's change it into the standard indeterminate form i.e. ($\frac{0}{0}$ or $\frac{\pm \infty}{\pm\infty}$), so that we could use L'Hopital's rule to evaluate the limit. \\
Now,
\[\lim_{x\rightarrow \infty}\biggl(1+\frac{1}{x}\biggl)^x=\lim_{x\rightarrow\infty} e^{\ln(1+\frac{1}{x})^x} = \lim_{x\rightarrow \infty} e^{x \ln(1+\frac{1}{x})}=\lim_{x\rightarrow \infty} e^{\frac{\ln(1+\frac{1}{x})}{\frac{1}{x}}}=e^{\ds\lim_{x\rightarrow\infty}\frac{\ln ⁡(1+\frac{1}{x})}{\frac{1}{x}}}\]
Now the limit is in  $\frac{0}{0}$  form. Thus we can apply L'Hopital's rule
\begin{align*}e^{\ds\lim_{x\rightarrow\infty}\frac{(\ln ⁡(1+\frac{1}{x}))'}{(\frac{1}{x})'}}
&= e^{\ds\lim_{x\rightarrow\infty}\frac{⁡\frac{1}{(1+\frac{1}{x})}\cdot(\frac{1}{x})'}{(\frac{1}{x})'}}\\
&=e^{\ds\lim_{x\rightarrow\infty}\frac{1}{(1+\frac{1}{x})}}\\
&=e^{\ds\frac{1}{\left(1+\ds\lim_{x\rightarrow\infty}\frac{1}{x}\right)}}\\
&=e
\end{align*}
\end{proof}

\subsection{Using power series}

\begin{prop} The infinite series
\begin{align}
1 + \frac{1}{1!}+\frac{1}{2!}+\frac{1}{3!}+\frac{1}{4!}+\frac{1}{5!}+\cdots
\end{align}
converges to the number $e$.
\end{prop}

\begin{proof}
The power series expansion of $e^x$ is
\begin{align}
e^x=1+x+\frac{x^2}{2!}+\frac{x^3}{3!}+\cdots
\end{align}
At $x=1$,
\begin{align}
e=1+1+\frac{1}{2!}+\frac{1}{3!}+\cdots
\end{align}
\end{proof}

\section{Irrationality of $e$}
\begin{thm} The number $e$ is irrational.
\end{thm}
\begin{proof}
We now that
\[e=1+1+\frac{1}{2!}+\frac{1}{3!}+\cdots\]
Denote the partial sum
\[s_n=1+1+\frac{1}{2!}+\frac{1}{3!}+\cdots+\frac{1}{n!}\]
Clearly, the following inequality is true for any $n=1,2,3,4,\ldots$
\begin{align}
0<e-s_n &= \frac{1}{(n+1)!}+\frac{1}{(n+2)!}+\frac{1}{(n+3)!}+\cdots\\
        &= \frac{1}{(n+1)!}\biggl[1+\frac{1}{(n+2)}+\frac{1}{((n+3)(n+2)}+\cdots\biggl]     \label{st1}
\end{align}
Now replace every $n+k$ in the denominator by $n+1$  since $k>1$ we've $n+k>n+1$.
This implies us
\[\frac{1}{(n+k)}<\frac{1}{(n+1)}\]
Thus,
\begin{align*}
\frac{1}{(n+1)!} \biggl[1+\frac{1}{(n+2)} + \frac{1}{(n+3)(n+2)}+\cdots\biggl]&<\frac{1}{(n+1)!}\biggl[1+\frac{1}{(n+1)}+\frac{1}{(n+1)(n+1)}+\cdots\biggl]   \\
                                  &=\frac{1}{(n+1)!}\biggl[1+\frac{1}{(n+1)}+\frac{1}{(n+1)^2}+\cdots\biggl] \\
                                  &=\frac{1}{(n+1)!}\biggl[\frac{n+1}{n}\biggl]\tag{Geometric sum}\\
                                  &=\frac{1}{n!n}
\end{align*}
Thus we have
\begin{align}\label{sttt}
\frac{1}{(n+1)!} \biggl[1+\frac{1}{(n+2)} + \frac{1}{(n+3)(n+2)}+\cdots\biggl]<\frac{1}{n!n}
\end{align}
Now use (\ref{sttt}) and (\ref{st1}) to obtain
\begin{align}\label{atc}
0<n!(e-s_n )<\frac{1}{n},\qquad  \text{ for } n =1,2,3,\ldots
\end{align}
Assume $e$ is rational i.e.  $e=\frac{p}{q}$.  Where $p$ and $q$ are relatively prime (the fraction is in lowest term).\\
Now choose any $n>q$. Which implies $n>1$ so is  $\frac{1}{n}<1$.
Thus, (\ref{atc}) becomes
\[
0<n!(e-s_n )<\frac{1}{n}<1
\]
This shows that  $n!(e-s_n )$  is not an integer. But
\begin{align*}
n!(e-s_n )&=n!\biggl(\frac{p}{q}-s_n\biggl)\\
          &=n!\biggl(\frac{p}{q}\biggl)-n!(s_n)\\
          &=\underbrace{\biggl(\frac{n!}{q}\biggl)}_{\text{integer}}p-\underbrace{n!\biggl(1+1+\frac{1}{2!}+\frac{1}{3!}+
          \cdots+\frac{1}{n!}\biggl)}_{\text{integer}}
          \tag{$q|n!\because n>q$.}\\
          &= \text{integer }-\text{ integer}=\text{integer}.
\end{align*}
Meaning $n!(e-s_n )$  is an integer which is a contradiction! Hence our assumption $e$ is rational is wrong.\\
Therefore, $e$ is irrational.
\end{proof}

\begin{thebibliography}{9}

\bibitem{Sep}
[Robert Ellis]~
Calculus with Analytic Geometry.

\bibitem{Jun}
[Tom Apostol]~
Mathematical Analysis, 2nd ed.

\bibitem{amsshort}
[Walter Rudin]  ~
Principles of Mathematical Analysis.

\end{thebibliography}
\end{document}
